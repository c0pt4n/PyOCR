\documentclass[12pt]{article}
\usepackage[utf8]{inputenc}
\usepackage{graphicx}
\usepackage{enumitem}
\usepackage{hyperref}
\usepackage{xcolor}
\usepackage{listings}
\usepackage{geometry}
\usepackage{titlesec}

\geometry{a4paper, margin=1in}

\titleformat{\section}
{\normalfont\Large\bfseries}{\thesection}{1em}{}
\titleformat{\subsection}
{\normalfont\large\bfseries}{\thesubsection}{1em}{}

\hypersetup{
    colorlinks=true,
    linkcolor=blue,
    filecolor=magenta,
    urlcolor=cyan,
}

\lstset{
    basicstyle=\ttfamily\small,
    columns=flexible,
    breaklines=true
}

\begin{document}

\begin{titlepage}
    \centering
    \vspace*{1cm}
    {\Huge\textbf{PyOCR Project Overview}\par}
    \vspace{1.5cm}
    {\Large Python Optical Character Recognition\par}
    \vspace{2cm}
    {\large A GUI application for Tesseract OCR\par}
    \vspace{1.5cm}
    {\large\textbf{Team Members:}\par}
    \vspace{0.5cm}
    {\large
    \begin{enumerate}[label=\arabic*.]
        \item Omar Mohamed Mahmoud Ibrahim
        \item Mohamed Ahmed Mohamed Ali Shehata
        \item Mohamed Ahmed Mohamed Atta
        \item Omar Mohamed Abdo
        \item Salah Eldin Mohamed Salah
        \item Hussein Arafat
    \end{enumerate}\par}
    \vfill
    {\large \today\par}
\end{titlepage}

\tableofcontents
\newpage

\section{Project Description}
PyOCR (Python Optical Character Recognition) is a desktop application that provides a graphical user interface for the Tesseract OCR engine. The application allows users to extract text from images using optical character recognition technology with an intuitive and user-friendly interface built with PyQt.

\section{Project Goals}
\begin{enumerate}
    \item Create a cross-platform GUI application for Tesseract OCR
    \item Implement image preprocessing capabilities to improve OCR accuracy
    \item Support multiple output formats
    \item Provide a user-friendly interface for OCR operations
    \item Ensure cross-platform compatibility (Windows, macOS, Linux)
\end{enumerate}

\section{Key Features}

\subsection{GUI Interface}
\begin{itemize}
    \item Modern PyQt-based user interface
    \item Responsive design that adapts to different screen sizes
    \item Threading implementation to prevent UI freezing during processing
    \item Progress indicators for OCR operations
\end{itemize}

\subsection{File Handling}
\begin{itemize}
    \item Support for multiple image formats (PNG, JPEG, TIFF)
    \item Automatic format conversion for unsupported formats
    \item File preview functionality
    \item File size limitations and validation
    \item Cross-platform file path handling
    \item OS-specific file system handling for Windows, macOS, and Linux
\end{itemize}

\subsection{Text Output Options}
\begin{itemize}
    \item Multiple output formats:
    \begin{itemize}
        \item Plain text
        \item hOCR (HTML)
        \item PDF
        \item Invisible-text-only PDF
        \item TSV
        \item ALTO
        \item PAGE
    \end{itemize}
    \item File saving with proper extension handling
    \item Overwrite confirmation dialog
    \item Auto-save functionality
    \item Recent files list
    \item Format-specific preview capabilities
    \item Proper encoding handling for text outputs
    \item Format validation for each output type
    \item Clipboard functionality:
    \begin{itemize}
        \item Copy/paste buttons with keyboard shortcuts
        \item Partial text selection copying
        \item Auto-copy option after OCR completion
        \item Clipboard history for recent OCR results
        \item Paste from clipboard for image processing
    \end{itemize}
\end{itemize}

\subsection{Image Processing}
\begin{itemize}
    \item Basic preprocessing pipeline using OpenCV:
    \begin{itemize}
        \item Grayscale conversion
        \item Noise reduction
        \item Contrast enhancement
        \item Thresholding
    \end{itemize}
    \item Document-specific preprocessing:
    \begin{itemize}
        \item Edge detection
        \item Perspective correction
        \item Border removal
        \item Shadow elimination
    \end{itemize}
    \item AI-enhanced image quality improvement:
    \begin{itemize}
        \item Transfer learning models
        \item Dataset for fine-tuning
        \item Model training pipeline
        \item Inference pipeline
        \item Comparison view of before/after enhancement
    \end{itemize}
    \item Performance optimization:
    \begin{itemize}
        \item Caching mechanisms
        \item Batch processing capabilities
        \item Processing priority settings
        \item Memory usage optimization for large files
    \end{itemize}
\end{itemize}

\section{Technical Architecture}

\subsection{Core Components}
\begin{enumerate}
    \item \textbf{Main Application}: Handles the application lifecycle and coordinates between components
    \item \textbf{UI Layer}: PyQt-based user interface components
    \item \textbf{OCR Engine}: Integration with Tesseract OCR via pytesseract
    \item \textbf{Image Processing}: OpenCV-based preprocessing pipeline
    \item \textbf{File I/O}: Handles file reading/writing operations
\end{enumerate}

\subsection{Dependencies}
\begin{itemize}
    \item PyQt6: UI framework
    \item pytesseract: Python wrapper for Tesseract OCR
    \item OpenCV (cv2): Image processing library
    \item Pillow: Python Imaging Library
    \item NumPy: Numerical computing library
    \item Matplotlib: Visualization library (for debugging and development)
\end{itemize}

\subsection{Directory Structure}
\begin{lstlisting}
PyOCR/
|-- docs/               # Documentation
|   |-- main/          # Project overview and main documentation
|   |-- img_enhance/   # Documentation for image enhancement module
|   |-- tests/         # Test documentation
|   `-- pdfs/          # PDF documentation
|-- src/               # Source code
|   |-- img_enhance/   # Image enhancement module
|   |-- main.py        # Main entry point
|   `-- mainui.py      # Main UI implementation
|-- tests/             # Test files
|-- requirements.txt   # Project dependencies
`-- README.md          # Project README
\end{lstlisting}

\section{Implementation Details}

\subsection{OCR Integration}
The application integrates with Tesseract OCR through the pytesseract library. The OCR process is executed in a separate thread to prevent UI freezing during processing, with progress updates shown to the user. The integration includes:
\begin{itemize}
    \item Setup of project environment \& dependencies installation
    \item Configuration of tesseract engine paths for cross-platform compatibility
    \item Error handling and recovery mechanisms
\end{itemize}

\subsection{Image Enhancement}
The image enhancement module provides various preprocessing techniques to improve OCR accuracy:
\begin{enumerate}
    \item \textbf{Basic Processing}: Grayscale conversion, noise reduction, contrast enhancement, thresholding
    \item \textbf{Advanced Processing}: Edge detection, perspective correction, border removal, shadow elimination
    \item \textbf{AI-Enhanced Processing}: Transfer learning models for image quality improvement
\end{enumerate}

\subsection{Cross-Platform Compatibility}
The application is designed to work across different operating systems:
\begin{itemize}
    \item Windows file path handling
    \item macOS file path handling
    \item Linux file path handling
    \item Cross-platform compatibility testing
\end{itemize}

\subsection{User Interface}
The UI is built with PyQt6 and includes:
\begin{itemize}
    \item File selection dialog
    \item Image preview area
    \item Text output display
    \item Format selection options
    \item Processing options
    \item Progress indicators
    \item Error messaging system for invalid files
\end{itemize}

\section{Development Roadmap}

\subsection{Phase 1: Basic Functionality}
\begin{itemize}
    \item Project setup and environment configuration
    \item Basic PyQt window structure with responsive design
    \item Tesseract integration
    \item Simple file selection and OCR processing
    \item Basic error handling
\end{itemize}

\subsection{Phase 2: Enhanced Features}
\begin{itemize}
    \item Image preprocessing pipeline
    \item Multiple output formats
    \item Clipboard functionality
    \item File saving options
    \item Format-specific preview capabilities
\end{itemize}

\subsection{Phase 3: Advanced Features}
\begin{itemize}
    \item AI-enhanced image processing
    \item Batch processing capabilities
    \item Advanced document preprocessing
    \item User preference saving
    \item Recent files list
    \item Clipboard history
\end{itemize}

\subsection{Phase 4: Optimization and Refinement}
\begin{itemize}
    \item Performance optimization
    \item UI/UX improvements
    \item Cross-platform testing and fixes
    \item Documentation completion
    \item Memory usage optimization
    \item Caching mechanisms
\end{itemize}

\section{Documentation Plan}

\subsection{Installation Documentation}
\begin{itemize}
    \item Windows installation guide
    \item macOS installation guide
    \item Linux installation guide
    \item Dependency requirements
\end{itemize}

\subsection{User Manual}
\begin{itemize}
    \item Interface overview
    \item Workflow tutorials
    \item Troubleshooting guide
    \item FAQ section
\end{itemize}

\subsection{Technical Documentation}
\begin{itemize}
    \item API documentation
    \item Class diagrams
    \item Implementation details
    \item Extension guidelines
\end{itemize}

\subsection{Example Documentation}
\begin{itemize}
    \item Sample use cases
    \item Batch processing examples
    \item Advanced configuration options
    \item Performance optimization tips
\end{itemize}

\section{Current Status and Future Work}

\subsection{Implemented Features}
\begin{itemize}
    \item Basic PyQt window structure
    \item Tesseract integration via pytesseract
    \item Simple file selection
    \item Basic OCR processing
    \item Text output display
\end{itemize}

\subsection{Planned Features}
\begin{itemize}
    \item Advanced image preprocessing
    \item AI-enhanced image quality improvement
    \item Multiple output formats support
    \item Cross-platform file path handling
    \item Batch processing capabilities
    \item Clipboard history
    \item Format-specific preview capabilities
    \item Performance optimization
\end{itemize}

\end{document} 