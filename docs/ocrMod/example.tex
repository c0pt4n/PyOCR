\section{Example Usage (example.py)}
\seclabel{example}

The \texttt{example.py} module provides a complete example script demonstrating how to use the OCR module from the command line. It serves as both a practical example and a simple command-line tool for OCR tasks.

\subsection{Script Overview}

The example script allows you to process an image file using the OCR engine with various options:

\begin{lstlisting}[language=bash]
python -m src.ocrMod.example image_file.jpg [output_format] [language]
\end{lstlisting}

\subsection{Command-Line Arguments}

\begin{itemize}
  \item \textbf{image\_path}: Path to the image file to process (required)
  \item \textbf{output\_format}: Output format (optional, default: 'text')
  \item \textbf{language}: OCR language(s) (optional, default: 'eng')
\end{itemize}

\subsection{Supported Output Formats}

The script supports all output formats provided by the OCR engine:
\begin{itemize}
  \item \textbf{text}: Plain text output
  \item \textbf{hocr}: HTML output with layout information
  \item \textbf{pdf}: PDF output with embedded text
  \item \textbf{tsv}: Tab-separated values with detailed information
  \item \textbf{alto}: ALTO XML format
  \item \textbf{page}: PAGE XML format
\end{itemize}

\subsection{Implementation Details}

The example script performs the following steps:

\begin{enumerate}
  \item Parses command-line arguments
  \item Validates the input file
  \item Initializes the OCR engine
  \item Processes the image with the specified parameters
  \item Handles the output based on the format:
    \begin{itemize}
      \item For text formats: Prints the result to the console
      \item For binary formats (PDF, hOCR, ALTO): Saves to a file
      \item For data formats (TSV): Prints a preview of the data
    \end{itemize}
\end{enumerate}

\subsection{Usage Examples}

\subsubsection{Basic Text Extraction}

\begin{lstlisting}[language=bash]
python -m src.ocrMod.example scan.jpg
\end{lstlisting}

This command extracts text from \texttt{scan.jpg} using the default settings (English language, text output) and prints the result to the console.

\subsubsection{Using Different Output Formats}

\begin{lstlisting}[language=bash]
python -m src.ocrMod.example scan.jpg hocr
\end{lstlisting}

This command extracts text from \texttt{scan.jpg} in hOCR format and saves it to \texttt{output.hocr}.

\begin{lstlisting}[language=bash]
python -m src.ocrMod.example scan.jpg pdf
\end{lstlisting}

This command creates a searchable PDF from \texttt{scan.jpg} and saves it to \texttt{output.pdf}.

\subsubsection{Using Different Languages}

\begin{lstlisting}[language=bash]
python -m src.ocrMod.example german_document.jpg text deu
\end{lstlisting}

This command extracts text from \texttt{german\_document.jpg} using the German language model.

\begin{lstlisting}[language=bash]
python -m src.ocrMod.example multilingual.jpg text eng+fra
\end{lstlisting}

This command extracts text from \texttt{multilingual.jpg} using both English and French language models.

\subsection{Extending the Example}

The example script can be extended in various ways:

\begin{itemize}
  \item Adding support for additional command-line options
  \item Implementing batch processing for multiple files
  \item Adding image preprocessing options
  \item Integrating with the image enhancement module
\end{itemize}

\subsection{Code Structure}

The example script is structured as follows:

\begin{lstlisting}[language=Python]
def main():
    # Parse command-line arguments
    # Validate input file
    # Initialize OCR engine
    # Process the image
    # Handle output based on format
    
if __name__ == "__main__":
    sys.exit(main())
\end{lstlisting}

This structure follows best practices for Python command-line scripts, making it easy to understand and extend. 