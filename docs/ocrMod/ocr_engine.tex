\section{OCR Engine Module (ocr\_engine.py)}
\seclabel{ocr_engine}

The \texttt{ocr\_engine.py} module provides the core functionality for text extraction using the Tesseract OCR engine. It encapsulates the Tesseract functionality in a simple, easy-to-use Python class.

\subsection{OCREngine Class}

\begin{lstlisting}[language=Python]
class OCREngine:
    def __init__(self, tesseract_cmd: Optional[str] = None):
        ...
\end{lstlisting}

The \texttt{OCREngine} class is the main interface to the Tesseract OCR engine. It provides methods for extracting text from images in various formats.

\subsubsection{Constructor}

\begin{lstlisting}[language=Python]
def __init__(self, tesseract_cmd: Optional[str] = None):
\end{lstlisting}

Initializes the OCR engine.

\begin{itemize}
  \item \textbf{tesseract\_cmd}: Path to the tesseract executable. If None, uses the default path.
\end{itemize}

\subsubsection{process\_image Method}

\begin{lstlisting}[language=Python]
def process_image(self, 
                  image_path: str, 
                  lang: str = 'eng',
                  output_format: str = 'text',
                  config: str = '') -> Union[str, Dict]:
\end{lstlisting}

Processes an image file and extracts text using Tesseract OCR.

\begin{itemize}
  \item \textbf{image\_path}: Path to the image file
  \item \textbf{lang}: Language(s) to use for OCR (default: 'eng')
  \item \textbf{output\_format}: Output format ('text', 'hocr', 'pdf', 'tsv', 'alto', 'page')
  \item \textbf{config}: Additional configuration parameters for Tesseract
  \item \textbf{Returns}: Extracted text or data in the specified format
\end{itemize}

\subsubsection{process\_pil\_image Method}

\begin{lstlisting}[language=Python]
def process_pil_image(self,
                     image: Image.Image,
                     lang: str = 'eng',
                     output_format: str = 'text',
                     config: str = '') -> Union[str, Dict]:
\end{lstlisting}

Processes a PIL Image object directly and extracts text using Tesseract OCR.

\begin{itemize}
  \item \textbf{image}: PIL Image object to process
  \item \textbf{lang}: Language(s) to use for OCR (default: 'eng')
  \item \textbf{output\_format}: Output format ('text', 'hocr', 'pdf', 'tsv', 'alto', 'page')
  \item \textbf{config}: Additional configuration parameters for Tesseract
  \item \textbf{Returns}: Extracted text or data in the specified format
\end{itemize}

\subsubsection{\_extract\_text Method}

\begin{lstlisting}[language=Python]
def _extract_text(self, 
                 image: Image.Image, 
                 lang: str, 
                 output_format: str, 
                 config: str) -> Union[str, Dict]:
\end{lstlisting}

Internal method that extracts text from an image using the appropriate Tesseract method based on the requested output format.

\begin{itemize}
  \item \textbf{image}: PIL Image object
  \item \textbf{lang}: Language(s) for OCR
  \item \textbf{output\_format}: Output format
  \item \textbf{config}: Additional Tesseract configuration
  \item \textbf{Returns}: Extracted text or data in the specified format
\end{itemize}

\subsection{Supported Output Formats}

The OCR engine supports the following output formats:

\begin{itemize}
  \item \textbf{text}: Plain text output
  \item \textbf{hocr}: HTML output with layout information
  \item \textbf{pdf}: PDF output with embedded text
  \item \textbf{tsv}: Tab-separated values with detailed information
  \item \textbf{alto}: ALTO XML format
  \item \textbf{page}: PAGE XML format
\end{itemize}

\subsection{Language Support}

The language parameter (\texttt{lang}) can be set to any language supported by Tesseract. Multiple languages can be specified by joining them with a plus sign, e.g., \texttt{'eng+fra'} for English and French.

To use a language, you need to have the corresponding language data files installed for Tesseract. The default language is English (\texttt{'eng'}).

\subsection{Configuration Options}

The \texttt{config} parameter allows you to pass additional configuration options to Tesseract. Some common options include:

\begin{itemize}
  \item \texttt{--psm N}: Page segmentation mode (0-13)
  \item \texttt{--oem N}: OCR Engine mode (0-3)
  \item \texttt{-c x=y}: Set parameter x to y
\end{itemize}

\subsubsection{Page Segmentation Modes (PSM)}

\begin{itemize}
  \item \texttt{--psm 0}: Orientation and script detection only
  \item \texttt{--psm 1}: Automatic page segmentation with OSD
  \item \texttt{--psm 3}: Fully automatic page segmentation, but no OSD (default)
  \item \texttt{--psm 6}: Assume a single uniform block of text
  \item \texttt{--psm 7}: Treat the image as a single text line
  \item \texttt{--psm 8}: Treat the image as a single word
  \item \texttt{--psm 10}: Treat the image as a single character
\end{itemize}

\subsection{Usage Examples}

\subsubsection{Basic Text Extraction}

\begin{lstlisting}[language=Python]
from src.ocrMod.ocr_engine import OCREngine

# Initialize the OCR engine
ocr = OCREngine()

# Extract text from an image
text = ocr.process_image("scan.jpg")
print(text)
\end{lstlisting}

\subsubsection{Using Different Languages}

\begin{lstlisting}[language=Python]
# Extract text in German
text_de = ocr.process_image("german_document.jpg", lang="deu")

# Extract text in multiple languages (English and Spanish)
text_multi = ocr.process_image("multilingual.jpg", lang="eng+spa")
\end{lstlisting}

\subsubsection{Using Different Output Formats}

\begin{lstlisting}[language=Python]
# Get hOCR output (HTML with layout information)
hocr = ocr.process_image("document.jpg", output_format="hocr")

# Get TSV output (detailed information)
tsv = ocr.process_image("document.jpg", output_format="tsv")
\end{lstlisting}

\subsubsection{Using Custom Configuration}

\begin{lstlisting}[language=Python]
# Use page segmentation mode 6 (single uniform block of text)
text = ocr.process_image("document.jpg", config="--psm 6")

# Use multiple configuration options
text = ocr.process_image(
    "document.jpg",
    config="--psm 6 --oem 1 -c tessedit_char_whitelist=0123456789"
)
\end{lstlisting}

\subsubsection{Processing a PIL Image}

\begin{lstlisting}[language=Python]
from PIL import Image
from src.ocrMod.ocr_engine import OCREngine

# Open an image with PIL
image = Image.open("document.jpg")

# Preprocess the image if needed
# image = image.convert('L')  # Convert to grayscale
# image = image.resize((image.width * 2, image.height * 2))  # Resize

# Initialize the OCR engine
ocr = OCREngine()

# Process the PIL image
text = ocr.process_pil_image(image)
print(text)
\end{lstlisting} 