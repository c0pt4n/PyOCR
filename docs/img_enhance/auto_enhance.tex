\section{auto\_enhance.py - Automated Image Enhancement}\seclabel{auto_enhance}\modulelabel{auto_enhance.py}

\subsection{Overview}
The \texttt{auto\_enhance.py} module provides functionality for automatically enhancing images for OCR without requiring manual parameter tuning. It uses image analysis techniques to determine optimal enhancement parameters based on image characteristics.

\subsection{Key Functions}

\subsubsection{analyze\_image()}\seclabel{analyze_image}
\begin{lstlisting}[language=Python, caption=Image Analysis Function]
def analyze_image(image: Union[str, Image.Image]) -> Dict[str, float]:
    """
    Analyze image and extract key metrics
    
    Args:
        image: PIL Image or path to image file
        
    Returns:
        Dictionary with image metrics
    """
\end{lstlisting}

This function analyzes an image and extracts metrics that are used to determine optimal enhancement parameters. It calculates:

\begin{itemize}
    \item \textbf{brightness}: Average brightness of the image (0-1)
    \item \textbf{contrast}: Standard deviation of pixel values, indicating contrast level (0-1)
    \item \textbf{sharpness}: Edge detection-based metric for image sharpness (0-1)
    \item \textbf{color\_saturation}: Color variation in the image (0-1)
    \item \textbf{noise\_level}: Estimated noise in the image (0-1)
    \item \textbf{might\_be\_skewed}: Boolean indicating if text appears skewed
    \item \textbf{r\_avg, g\_avg, b\_avg}: Average values of RGB channels (0-1)
\end{itemize}

\subsubsection{determine\_optimal\_params()}\seclabel{determine_optimal_params}
\begin{lstlisting}[language=Python, caption=Parameter Determination Function]
def determine_optimal_params(metrics: Dict[str, float]) -> EnhancementParams:
    """
    Determine optimal enhancement parameters based on image metrics
    
    Args:
        metrics: Dictionary with image metrics from analyze_image
        
    Returns:
        EnhancementParams with suggested values
    """
\end{lstlisting}

This function uses the metrics from \hyperref[sec:analyze_image]{\texttt{analyze\_image()}} to determine optimal enhancement parameters. It implements a rule-based system that adjusts parameters based on image characteristics:

\begin{itemize}
    \item Adjusts brightness if the image is too dark or too bright
    \item Increases contrast for low-contrast images
    \item Enhances sharpness for blurry images
    \item Increases color saturation for faded documents
    \item Enables denoising for noisy images
    \item Enables deskewing for skewed text
    \item Determines if binarization would be beneficial based on content
\end{itemize}

The function returns an \hyperref[sec:enhancement_params]{\texttt{EnhancementParams}} object with the optimized settings.

\subsubsection{auto\_enhance\_image()}\seclabel{auto_enhance_function}
\begin{lstlisting}[language=Python, caption=Auto Enhancement Function]
def auto_enhance_image(image: Union[str, Image.Image], 
                      preview: bool = False, 
                      preview_size: Tuple[int, int] = (400, 400)) -> Image.Image:
    """
    Automatically enhance an image for OCR by analyzing it and applying optimal enhancements
    
    Args:
        image: PIL Image or path to image file
        preview: If True, returns a smaller preview image
        preview_size: Size of preview if preview=True
        
    Returns:
        Enhanced PIL Image
    """
\end{lstlisting}

This is the main entry point for automatic enhancement. It:
\begin{enumerate}
    \item Analyzes the image using \hyperref[sec:analyze_image]{\texttt{analyze\_image()}}
    \item Determines optimal parameters using \hyperref[sec:determine_optimal_params]{\texttt{determine\_optimal\_params()}}
    \item Creates an \hyperref[sec:enhancer_class]{\texttt{ImageEnhancer}} with those parameters
    \item Applies the enhancements and returns the enhanced image
\end{enumerate}

\subsubsection{batch\_auto\_enhance()}\seclabel{batch_enhance}
\begin{lstlisting}[language=Python, caption=Batch Enhancement Function]
def batch_auto_enhance(image_paths: List[str], output_dir: str) -> List[str]:
    """
    Automatically enhance multiple images and save them
    
    Args:
        image_paths: List of paths to images
        output_dir: Directory to save enhanced images
        
    Returns:
        List of paths to enhanced images
    """
\end{lstlisting}

This function processes multiple images in batch mode, saving the enhanced versions to the specified output directory. It internally uses \hyperref[sec:auto_enhance_function]{\texttt{auto\_enhance\_image()}} for each image.

\subsubsection{get\_enhancement\_preview\_grid()}\seclabel{preview_grid}
\begin{lstlisting}[language=Python, caption=Preview Grid Function]
def get_enhancement_preview_grid(image: Union[str, Image.Image], 
                                variations: Optional[List[Dict[str, float]]] = None) -> Image.Image:
    """
    Generate a grid of enhancement previews with different parameters
    
    Args:
        image: PIL Image or path to image file
        variations: List of dictionaries with parameter variations
        
    Returns:
        Grid image with different enhancement options
    """
\end{lstlisting}

This function creates a visual grid of enhancement options, allowing users to compare different parameter settings. The grid includes:
\begin{itemize}
    \item The original image
    \item Multiple variations with different enhancement parameters
\end{itemize}

By default, it includes the following variations:
\begin{itemize}
    \item Original
    \item Brighter with higher contrast
    \item Higher contrast and sharper
    \item Darker with higher contrast
    \item Binarized (black and white)
    \item Denoised and sharper
\end{itemize}

\subsection{Implementation Notes}
\begin{itemize}
    \item The automatic enhancement uses a heuristic approach based on empirical rules
    \item Image analysis focuses on characteristics that impact OCR quality (brightness, contrast, etc.)
    \item The algorithm is conservative to avoid over-processing images in ways that might harm OCR
    \item The preview grid function allows users to visually compare options and select the best approach
\end{itemize}

\subsection{Usage Example}
\begin{lstlisting}[language=Python, caption=Example of Auto Enhancement]
from PIL import Image
from src.img_enhance.auto_enhance import auto_enhance_image, get_enhancement_preview_grid

# Automatically enhance an image
input_path = "document.jpg"
enhanced = auto_enhance_image(input_path)
enhanced.save("enhanced_document.jpg")

# Generate a preview grid of different enhancement options
grid = get_enhancement_preview_grid(input_path)
grid.save("enhancement_options.jpg")
\end{lstlisting}

\subsection{Related Modules}
\begin{itemize}
    \item This module depends on the \hyperref[sec:enhancer]{\texttt{enhancer.py}} module for the enhancement functionality
    \item The \hyperref[sec:text_presets]{\texttt{text\_presets.py}} module provides an alternative approach with pre-defined presets
\end{itemize} 