\section{presets\_cli.py - Command Line Interface for Text Presets}\seclabel{presets_cli}\modulelabel{presets_cli.py}

\subsection{Overview}
The \texttt{presets\_cli.py} module provides a command-line interface for applying the specialized text enhancement presets to images. It allows users to quickly enhance images for OCR without programming knowledge, compare results, and save the enhanced images.

\subsection{Command Line Arguments}\seclabel{presets_cli_args}

\begin{lstlisting}[language=Python, caption=Command Line Argument Parser]
def parse_args():
    """Parse command-line arguments"""
    parser = argparse.ArgumentParser(description='Image Enhancement Presets for OCR Text')
    
    parser.add_argument('input_path', help='Input image path')
    parser.add_argument('--output-path', '-o', help='Output image path')
    parser.add_argument('--preset', '-p', type=int, default=0, 
                        help='Enhancement preset: 0=Mixed Content, 1=Document, 2=Text Only, 3=Receipt')
    parser.add_argument('--compare', '-c', action='store_true', 
                        help='Create comparison image')
    
    return parser.parse_args()
\end{lstlisting}

The command-line interface accepts the following arguments:

\begin{itemize}
    \item \textbf{input\_path} (required): Path to the input image file
    \item \textbf{-{}-output-path, -o} (optional): Path where the enhanced image should be saved
    \item \textbf{-{}-preset, -p} (optional): Preset ID to apply (default: 0)
        \begin{itemize}
            \item 0 = \hyperref[sec:mixed_content_preset]{Mixed Content with Text Enhancement}
            \item 1 = \hyperref[sec:text_document_preset]{Document Text Enhancement}
            \item 2 = \hyperref[sec:text_only_preset]{Pure Text Enhancement}
            \item 3 = \hyperref[sec:receipt_preset]{Receipt Enhancement}
        \end{itemize}
    \item \textbf{-{}-compare, -c} (optional): Create a side-by-side comparison of original and enhanced images
\end{itemize}

\subsection{Main Process Flow}\seclabel{presets_cli_flow}

\begin{lstlisting}[language=Python, caption=Main Process Function]
def main():
    """Main entry point"""
    args = parse_args()
    
    # Check if input file exists
    if not os.path.exists(args.input_path):
        print(f"Error: Input file '{args.input_path}' not found.")
        return 1
    
    # Get preset name
    preset_name = get_preset_name(args.preset)
    print(f"Applying preset: {preset_name}")
    
    # Load original image
    original = Image.open(args.input_path).convert('RGB')
    
    # Apply enhancement
    enhanced = enhance_with_preset(original, args.preset)
    
    # Determine output path
    if args.output_path:
        output_path = args.output_path
    else:
        # Use input filename with preset suffix
        dirname = os.path.dirname(args.input_path)
        filename = os.path.basename(args.input_path)
        base, ext = os.path.splitext(filename)
        output_path = os.path.join(dirname, f"{base}_preset{args.preset}{ext}")
    
    # Create output directory if needed
    os.makedirs(os.path.dirname(output_path) or '.', exist_ok=True)
    
    # Save enhanced image
    enhanced.save(output_path)
    print(f"Saved enhanced image to: {output_path}")
    
    # Create comparison if requested
    if args.compare:
        comparison_path = os.path.splitext(output_path)[0] + "_comparison.png"
        comparison = create_comparison_image(
            original, 
            enhanced,
            title=f"Preset: {preset_name}"
        )
        comparison.save(comparison_path)
        print(f"Saved comparison image to: {comparison_path}")
    
    return 0
\end{lstlisting}

The main process:

\begin{enumerate}
    \item Parses command-line arguments using \hyperref[sec:presets_cli_args]{\texttt{parse\_args()}}
    \item Verifies the input file exists
    \item Gets the preset name using \hyperref[sec:get_preset_name]{\texttt{get\_preset\_name()}}
    \item Loads the original image
    \item Applies the selected preset using \hyperref[sec:enhance_with_preset]{\texttt{enhance\_with\_preset()}}
    \item Determines the output path (user-specified or auto-generated)
    \item Creates the output directory if needed
    \item Saves the enhanced image
    \item Optionally creates and saves a comparison image
\end{enumerate}

\subsection{Output Naming Convention}\seclabel{naming_convention}
If no output path is specified, the tool automatically generates an output filename using the following pattern:
\begin{verbatim}
[original_filename]_preset[preset_id].[original_extension]
\end{verbatim}

For example, if the input file is \texttt{document.jpg} and preset 2 is applied, the output will be \texttt{document\_preset2.jpg}.

If comparison is requested, the comparison image will be saved as:
\begin{verbatim}
[output_filename_without_extension]_comparison.png
\end{verbatim}

For example: \texttt{document\_preset2\_comparison.png}

\subsection{Usage Examples}\seclabel{presets_cli_examples}

\subsubsection{Basic Usage}
\begin{lstlisting}[language=Bash, caption=Basic CLI Usage]
# Apply the default mixed content preset
python src/img_enhance/presets_cli.py document.jpg

# Apply the receipt preset
python src/img_enhance/presets_cli.py receipt.jpg --preset 3

# Specify output file
python src/img_enhance/presets_cli.py document.jpg -o enhanced_document.jpg

# Create comparison image
python src/img_enhance/presets_cli.py document.jpg --compare
\end{lstlisting}

\subsubsection{Batch Processing}
While not directly supported by the presets\_cli.py script, batch processing can be achieved using shell scripts:

\begin{lstlisting}[language=Bash, caption=Batch Processing Example]
# Process all JPG files in a directory with the Document preset
for file in *.jpg; do
    python src/img_enhance/presets_cli.py "$file" --preset 1 --compare
done
\end{lstlisting}

\subsection{Integration with Other Tools}\seclabel{integration}
The presets\_cli.py script can be easily integrated with other command-line tools and workflows:

\begin{lstlisting}[language=Bash, caption=Integration Example]
# Enhance an image and pass to OCR
python src/img_enhance/presets_cli.py document.jpg -o enhanced.jpg
tesseract enhanced.jpg output -l eng

# Process a scanned image
scan-image | python src/img_enhance/presets_cli.py - -o enhanced.jpg
\end{lstlisting}

\subsection{Implementation Notes}
\begin{itemize}
    \item The script uses the \hyperref[sec:text_presets]{\texttt{text\_presets.py}} module for the actual enhancement operations
    \item It leverages the comparison functionality from utils.py to generate side-by-side comparisons
    \item Error handling ensures the script exits gracefully if the input file doesn't exist
    \item Output directories are created automatically if they don't exist
\end{itemize}

\subsection{Related Modules}
\begin{itemize}
    \item \hyperref[sec:text_presets]{\texttt{text\_presets.py}} - Provides the preset enhancement functions
    \item \hyperref[sec:enhancer]{\texttt{enhancer.py}} - Core enhancement functionality
    \item \hyperref[sec:auto_enhance]{\texttt{auto\_enhance.py}} - Alternative approach with automatic parameter determination
\end{itemize} 