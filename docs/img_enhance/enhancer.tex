\section{enhancer.py - Core Enhancement Functionality}\seclabel{enhancer}\modulelabel{enhancer.py}

\subsection{Overview}
The \texttt{enhancer.py} module contains the core functionality for image enhancement in the PyOCR system. It provides parameter definition and enhancement operations through two main classes: \texttt{EnhancementParams} and \texttt{ImageEnhancer}.

\subsection{EnhancementParams Class}\seclabel{enhancement_params}
\begin{lstlisting}[language=Python, caption=EnhancementParams Class Definition]
@dataclass
class EnhancementParams:
    """Parameters for image enhancement"""
    brightness: float = 1.0  # 0.5 to 1.5
    contrast: float = 1.0    # 0.5 to 2.0
    sharpness: float = 1.0   # 0.0 to 2.0
    color: float = 1.0       # 0.0 to 2.0
    denoise: bool = False
    binarize: bool = False
    binarize_threshold: int = 128  # 0 to 255
    deskew: bool = False
    resize_factor: Optional[float] = None
\end{lstlisting}

The \texttt{EnhancementParams} class is a dataclass that encapsulates all parameters used for image enhancement:

\begin{itemize}
    \item \textbf{brightness} (float, default=1.0): Controls the overall brightness of the image. Values > 1.0 increase brightness, values < 1.0 decrease brightness. Range: 0.5-1.5.
    \item \textbf{contrast} (float, default=1.0): Controls the contrast between light and dark areas. Values > 1.0 increase contrast, values < 1.0 decrease contrast. Range: 0.5-2.0.
    \item \textbf{sharpness} (float, default=1.0): Controls edge definition. Higher values make text edges sharper. Range: 0.0-2.0.
    \item \textbf{color} (float, default=1.0): Controls color saturation. 0.0 removes all color (grayscale), values > 1.0 increase saturation. Range: 0.0-2.0.
    \item \textbf{denoise} (bool, default=False): When True, applies a noise reduction filter.
    \item \textbf{binarize} (bool, default=False): When True, converts the image to binary (black and white).
    \item \textbf{binarize\_threshold} (int, default=128): Threshold value for binarization. Range: 0-255.
    \item \textbf{deskew} (bool, default=False): When True, attempts to straighten text in the image.
    \item \textbf{resize\_factor} (Optional[float], default=None): When set, resizes the image by this factor (e.g., 1.5 = 150\%).
\end{itemize}

See \hyperref[sec:parameters]{Enhancement Parameters} for more details on the effects of these parameters.

The class also provides methods for converting parameters to and from dictionary format:
\begin{itemize}
    \item \texttt{to\_dict()}: Converts parameters to a dictionary
    \item \texttt{from\_dict(cls, params\_dict)}: Class method to create parameters from a dictionary
\end{itemize}

\subsection{ImageEnhancer Class}\seclabel{enhancer_class}
The \texttt{ImageEnhancer} class handles the actual image enhancement process:

\begin{lstlisting}[language=Python, caption=ImageEnhancer Class Definition]
class ImageEnhancer:
    """Class for enhancing images for OCR processing"""
    
    def __init__(self, params: Optional[EnhancementParams] = None):
        """Initialize with enhancement parameters"""
        self.params = params or EnhancementParams()
\end{lstlisting}

\subsubsection{Key Methods}
\begin{itemize}
    \item \texttt{set\_params(params)}: Updates all enhancement parameters at once
    \item \texttt{update\_params(**kwargs)}: Updates specific parameters by keyword arguments
    \item \texttt{enhance(image)}: Enhances an image using the current parameters
    \item \texttt{\_apply\_enhancements(img)}: Internal method that applies all enhancements in sequence
    \item \texttt{\_binarize\_image(img, threshold)}: Converts image to binary (black and white)
    \item \texttt{\_deskew\_image(img)}: Attempts to straighten text in the image
\end{itemize}

\subsection{Enhancement Process}\seclabel{enhancement_process}
When enhancing an image, the following steps are performed in sequence:

\begin{enumerate}
    \item Resize the image (if resize\_factor is specified)
    \item Apply brightness adjustment
    \item Apply contrast adjustment
    \item Apply color saturation adjustment
    \item Apply sharpness adjustment
    \item Apply denoising (if enabled)
    \item Apply deskewing (if enabled)
    \item Apply binarization (if enabled)
\end{enumerate}

The order of operations is important as each step can affect the results of subsequent steps.

\subsection{Implementation Details}
The enhancement functionality is built on top of the Python Imaging Library (PIL/Pillow), leveraging its built-in enhancement capabilities:

\begin{itemize}
    \item \texttt{ImageEnhance.Brightness} for brightness adjustment
    \item \texttt{ImageEnhance.Contrast} for contrast adjustment
    \item \texttt{ImageEnhance.Color} for color saturation adjustment
    \item \texttt{ImageEnhance.Sharpness} for sharpness adjustment
    \item \texttt{ImageFilter.MedianFilter} for noise reduction
\end{itemize}

\subsection{Usage Example}
\begin{lstlisting}[language=Python, caption=Example usage of ImageEnhancer]
from PIL import Image
from src.img_enhance.enhancer import ImageEnhancer, EnhancementParams

# Create custom parameters
params = EnhancementParams(
    brightness=1.2,
    contrast=1.5,
    sharpness=1.3,
    denoise=True
)

# Create enhancer with parameters
enhancer = ImageEnhancer(params)

# Load and enhance an image
image = Image.open("input.jpg")
enhanced = enhancer.enhance(image)
enhanced.save("output.jpg")

# Update parameters and re-enhance
enhancer.update_params(binarize=True, binarize_threshold=150)
binary_enhanced = enhancer.enhance(image)
binary_enhanced.save("binary_output.jpg")
\end{lstlisting}

\subsection{Related Modules}
\begin{itemize}
    \item The \hyperref[sec:auto_enhance]{auto\_enhance.py} module uses the \texttt{ImageEnhancer} class for automatic enhancement.
    \item The \hyperref[sec:text_presets]{text\_presets.py} module defines presets using the \texttt{EnhancementParams} class.
\end{itemize} 